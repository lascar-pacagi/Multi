\documentclass[a4paper]{article}

\usepackage[utf8]{inputenc}
\usepackage[T1]{fontenc}
\usepackage[frenchb]{babel}
\usepackage{aeguill}
\usepackage{graphicx}
\usepackage{listings}
\usepackage{textcomp}
\usepackage{tabularx}
\usepackage{minted}
%\usepackage{vaucanson}

\setlength{\textheight}{240mm}
\setlength{\textwidth}{162mm}
\setlength{\topmargin}{-15mm}
\setlength{\oddsidemargin}{+1mm}
\setlength{\unitlength}{1cm}


\newcommand{\code}[1]{\texttt{#1}}
\newtheorem{question}{Question}
\newtheorem{remarque}{Remarque}

\title{Architecture et système d'exploitation --- \\Bibliothèque de \emph{threads} partie 1}
\author{Pascal Garcia}
%\date{Année universitaire 2014--2015}
\lstset{language=[ANSI]C}

\begin{document}
\maketitle

\renewcommand{\labelitemi}{$\bullet$}
\renewcommand{\labelitemii}{$\star$}

\section{Création de \emph{threads}, ordonnancement et structures de données}

Nous allons présenter les fonctions permettant de créer des \emph{threads}, de les rendre prêts à s'exécuter et les fonctions
permettant de gérer l'ordonnancement entre les \emph{threads}.
Nous présenterons ensuite les structures de données associées à la gestion des \emph{threads}.

\subsection{Fonctions de création et de lancement}

\begin{itemize}

\item La fonction \mintinline{c}|int create_thread(long initial_address, int priority, char* name, int nb_args, ...);| 
(fichier \verb+thread.h+) permet de créer un \emph{thread}. On a~:
\begin{itemize}
\item \verb+initial_address+ est l'adresse de la fonction qui sera le code du \emph{thread}~;
\item \verb+priority+ est la priorité du \emph{thread}, plus cette valeur est grande et plus la priorité est importante~;
\item \verb+name+ est le nom du \emph{thread}~;
\item \verb+nb_args+ est le nombre de paramètres de la fonction représentant le code du \emph{thread}. On suppose que
tous les arguments sont de type \mintinline{c}|int|~;
\item la fonction retourne un entier représentant le numéro du \emph{thread} ou $-1$ si le \emph{thread} n'a pu être créé.
Le \emph{thread} créé est initialement dans l'état \textit{suspendu} (état \mintinline{c}|SUSPENDED|). Pour le rendre prêt
à être exécuté (état \mintinline{c}|READY|), il faut utiliser la fonction \mintinline{c}|resume| que nous décrivons ci-dessous.
\end{itemize}

\item La fonction \mintinline{c}|bool resume(int thread_id)| (fichier \verb+thread.h+)  permet de passer le \emph{thread} 
\mintinline{c}|thread_id| de l'état \mintinline{c}|SUSPENDED| à l'état \mintinline{c}|READY|.\\

\end{itemize}

\subsection{Exemple}

Nous donnons ci-dessous un exemple de programme que nous allons commenter~:
\begin{minted}{c}

#include <stdio.h>
#include "thread.h"

void fct(long n)
{
  while(1) {
    printf("%ld\n", n);
  }
}

int main()
{
  int t1 = create_thread((long)fct, DEFAULT_PRIORITY, "thread1", 1, 1);
  int t2 = create_thread((long)fct, DEFAULT_PRIORITY, "thread2", 1, 2);
  int t3 = create_thread((long)fct, DEFAULT_PRIORITY, "thread3", 1, 3);
  resume(t1);
  resume(t2);
  resume(t3);
  return 0;
}

\end{minted}

Ce programme va créer trois \emph{threads}, dont le code est la fonction \mintinline{c}|fct|. La fonction \mintinline{c}|fct| pour le \emph{thread} \mintinline{c}|t1|
aura comme paramètre la valeur \verb+1+, pour le \emph{thread} \mintinline{c}|t2| ce sera la valeur \verb+2+ et pour le \emph{thread} \verb+t3+ 
la valeur \verb+3+. Les trois \emph{threads} passent dans l'état \mintinline{c}|READY| grâce 
à la fonction \mintinline{c}|resume|. Notons que la fonction \mintinline{c}|main| est aussi un \emph{thread} qui est créé par une des fonctions d'initialisation de la bibliothèque.\\
Lorsque l'on exécutera ce programme, on verra apparaître à l'écran une succession de \verb+1+, \verb+2+ et \verb+3+. Nous 
allons décrire dans la sous-section suivante comment fonctionne l'ordonnancement entre les \emph{threads}.

\subsection{Ordonnancement}

Un \emph{thread} qui est prêt à s'exécuter (état \mintinline{c}|READY|) sera choisi par l'ordonnanceur selon sa priorité. Afin que tous les 
\emph{threads} (ayant la priorité la plus forte) puissent être servis de manière équitable, il est alloué un certain quantum de temps ($100ms$)
à chaque \emph{thread} avant que l'ordonnanceur ne soit appelé en vu de passer la main à un autre \emph{thread} de même priorité. La variable
globale \mintinline{c}|preempt| permet de connaître le temps d'exécution restant pour le \emph{thread} en cours d'exécution.\\
Toutes les $10ms$, le traitant d'interruption \mintinline{c}|clock_handler| est appelé. Son rôle est de compter le nombre
de pas de temps (par tranche de $10ms$) qui s'est écoulé depuis le début du programme (les interruptions liées au \emph{timer} sont activées
par une des fonctions d'initialisation de la bibliothèque) et d'appeler l'ordonnanceur quand le temps alloué au \emph{thread} courant
est terminé.\\

Le numéro du \emph{thread} courant est dans la variable globale \mintinline{c}|current_thread|. La liste globale \mintinline{c}|ready_list| contient
les numéros des \emph{threads} prêts à s'exécuter triés par ordre de priorité. Le \emph{thread} courant n'est pas dans la \mintinline{c}|ready_list|.
La fonction \mintinline{c}|bool reschedule()| permet de changer le \emph{thread} courant en passant la main à un des \emph{threads} dont le numéro est dans la \mintinline{c}|ready_list|. Cette fonction renvoie \mintinline{c}|true|
si on a effectivement changé de \emph{thread}. Elle renvoie \mintinline{c}|false| si aucun \emph{thread} de même priorité (ou supérieure) n'était présent dans la \mintinline{c}|ready_list|.
Notons que c'est cette fonction qui, entre autres, réinitialise la variable \mintinline{c}|preempt| pour que le \emph{thread} qui va prendre la main
puisse avoir son quantum de temps d'exécution.\\

La fonction \mintinline{c}|bool ready(int thread_id, bool resched)| est une fonction utilitaire qui permet de mettre dans la liste des prêts
(\mintinline{c}|ready_list|) le \emph{thread} \mintinline{c}|thread_id| et d'appeler ou non la fonction \mintinline{c}|reschedule| si le paramètre \mintinline{c}|resched| est à la valeur 
\mintinline{c}|true|. Cette fonction rend \mintinline{c}|true| si le numéro de \emph{thread} est correct et renvoie \mintinline{c}|false| sinon.\\

\textbf{Question 1~:} compléter le traitant d'interruptions \mintinline{c}|clock_handler| (fichier \verb+time.c+) afin de passer la main à un autre \emph{thread} lorsque le quantum de temps du \emph{thread} en cours est dépassé. Compiler et exécuter le programme décrit plus haut afin de le tester.\\

\subsection{Structures de données}

Le fichier \verb+thread.h+ décrit les structures de données liées aux \emph{threads}. Nous allons les passer en revue.

\begin{minted}{c}
#define DEFAULT_PRIORITY 10
\end{minted}

Cette constante est utilisable par l'utilisateur lorsqu'il crée un \emph{thread} pour indiquer une priorité par défaut.

\begin{minted}{c}
#define MAX_NB_THREAD 16
\end{minted}

Cette constante permet de fixer le nombre maximum de \emph{threads} que l'utilisateur pourra créer.

\begin{minted}{c}
#define STACK_SIZE 65536
\end{minted}

Cette constante définie la taille de la pile allouée à un \emph{thread}.

\begin{minted}{c}
#define NULL_THREAD 0
\end{minted}

Cette constante est le numéro d'un \emph{thread} spécial qui est créé à l'initialisation du système. Ce \emph{thread} prend la main au lancement
du système et lorsqu'aucun \emph{thread} n'est prêt à s'exécuter.

\begin{minted}{c}
#define is_bad_thread_id(x) (x <= 0 || x >= MAX_NB_THREAD)
\end{minted}

La \emph{macro} \mintinline{c}|is_bad_thread_id| permet de savoir si le numéro de \emph{thread} est un numéro valide. 

\begin{minted}{c}
#define NB_REGISTERS 8
\end{minted}

Cette constante définie le nombre de registres à sauvegarder lors d'un changement de contexte.

\begin{minted}{c}
enum {CURRENT, READY, WAITING, JOIN, ASLEEP, SUSPENDED, RECEIVE, FREE};
\end{minted}

Cette énumération représente les différents états dans lesquels peut se trouver un \emph{thread}~:\\
\begin{itemize}
\item \mintinline{c}|CURRENT|, le \emph{thread} est en cours d'exécution (il n'y a qu'un seul \emph{thread} dans cet état). Le numéro du
\emph{thread} est alors dans la variable \mintinline{c}|current_thread|~;\\
\item \mintinline{c}|READY|, le \emph{thread} est prêt à s'exécuter. Le numéro du \emph{thread} est alors dans la \mintinline{c}|ready_list|~;\\
\item \mintinline{c}|WAITING|, le \emph{thread} est bloqué dans la file d'attente d'un sémaphore (voir la section suivante)~;\\
\item \mintinline{c}|JOIN|, le \emph{thread} est en attente de la destruction d'un autre \emph{thread}~;\\
\item \mintinline{c}|ASLEEP|, le \emph{thread} est endormi et sera réveillé par le \mintinline{c}|clock_handler| lorsque le temps d'attente sera écoulé~;\\
\item \mintinline{c}|SUSPENDED|, le \emph{thread} est suspendu. Pour qu'il redevienne prêt à s'exécuter, il faut utiliser la méthode \mintinline{c}|resume|. C'est
l'état initial d'un \emph{thread} après sa création~;\\
\item \mintinline{c}|RECEIVE|, le \emph{thread} est dans l'attente d'un message~;\\
\item \mintinline{c}|FREE|, le \emph{thread} n'est pas alloué.\\
\end{itemize}

\begin{minted}{c}
typedef struct {
  int   state;
  int   priority;
  int   semaphore;
  char* name;     
  long  registers[NB_REGISTERS];
  long  stack[STACK_SIZE];
  list  join_list;
  int   join_thread;
  int   message;
  bool  received;
} thread;
\end{minted}

Cette structure représente le contexte d'un \emph{thread}~:\\
\begin{itemize}
\item \mintinline{c}|state|, l'état du \emph{thread}~;\\
\item \mintinline{c}|priority|, la priorité du \emph{thread} (plus ce nombre est élevé, plus la priorité est forte)~;\\
\item \mintinline{c}|semaphore|, le numéro de sémaphore dans la file duquel le \emph{thread} est bloqué ou $-1$ si le \emph{thread} n'est dans la
file d'attente d'aucun sémaphore. Ce numéro permet de retirer le \emph{thread} de la file 
d'attente du sémaphore lors de la destruction du \emph{thread}~;\\
\item \mintinline{c}|name|, le nom du \emph{thread}~;\\
\item \mintinline{c}|registers|, la valeur des registres pour sauvegarder l'état de la machine lors de la commutation de contexte et pour 
remettre correctement à jour les registres lorsque le \emph{thread} redevient actif~;\\
\item \mintinline{c}|stack|, la pile du \emph{thread}. En effet, les variables locales, les paramètres des fonctions et les adresses de retour de celles-ci doivent être sauvegardées lors de la commutation de contexte, et remises à jour correctement lorsque le \emph{thread} reprend la main~;\\
\item \mintinline{c}|join_list|, la liste des \emph{threads} qui sont en attente de la destruction de ce \emph{thread}. Lors de la destruction du \emph{thread},
les \emph{thread} contenus dans cette liste vont être mis dans
l'état \mintinline{c}|READY| et pourront donc de nouveau être exécutés~;\\
\item \mintinline{c}|join_thread|, le \emph{thread} dont on attend la destruction. Ceci nous permet de retirer le \emph{thread} de la liste \mintinline{c}|join_list|
du \emph{thread} \mintinline{c}|join_thread| lors de la destruction du \emph{thread} (sinon lorsque le \emph{thread} \mintinline{c}|join_thread| sera détruit il voudra
remettre dans l'état \mintinline{c}|READY| un \emph{thread} qui n'existe plus ou qui n'est pas le bon)~;\\
\item \mintinline{c}|message|, le message envoyé à ce \emph{thread} (dans cette implémentation, le message n'est qu'un entier)~;\\
\item \mintinline{c}|received|, indique si un message a été envoyé à ce \emph{thread}.\\
\end{itemize}

\begin{minted}{c}
extern thread thread_table[];
\end{minted}

Ce tableau contient les informations sur les \emph{threads}. Notons que le numéro d'un \emph{thread} correspond à son indice dans
ce tableau.

\begin{minted}{c}
extern int nb_thread;
\end{minted}

Cette variable contient le nombre actuel de \emph{threads} dans le système.

\begin{minted}{c}
extern int current_thread;
\end{minted}

Cette variable contient le numéro du \emph{thread} courant (en cours d'exécution).

\section{Atomicité}

Dans notre gestionnaire de \emph{threads}, il est important d'assurer l'atomicité de certaines opérations. Par exemple,
lors de la création d'un \emph{thread}, on va rechercher dans la table des \emph{threads} \mintinline{c}|thread_table| un \emph{thread} dont
l'état est \mintinline{c}|FREE|. Le code pourrait analyser les entrées de la table \mintinline{c}|thread_table| et se rendre compte que l'entrée $2$, par exemple,
est libre. S'il y a commutation de contexte à ce moment là, le \emph{thread} qui va prendre la main pourrait à son tour créer un autre \emph{thread}
et voir que l'emplacement $2$ est libre dans la table \mintinline{c}|thread_table|. La seule façon de perdre la main pour un \emph{thread} non volontairement
est d'être interrompu par une interruption. On dispose de trois fonctions permettant de masquer ou non les interruptions~:\\

\begin{itemize}
\item \mintinline{c}|void enable()|, autorise les interruptions~;\\

\item \mintinline{c}|status disable()|, désactive les interruptions et renvoie l'état actuel des interruptions (masquées ou non)~;\\

\item \mintinline{c}|void restore(status old)|, replace les interruptions dans l'état \mintinline{c}|old|.

\end{itemize}

\section{Synchronisation des \emph{threads}}

Nous allons présenter les fonctions permettant de synchroniser les \emph{threads}.

\subsection{Fonctions \emph{yield} et \emph{join}}

La procédure \mintinline{c}|void yield()| permet au \emph{thread} courant de passer la main à un autre \emph{thread} de priorité supérieure ou égale (s'il y en a un). La
fonction \mintinline{c}|bool join(int thread_id)| permet de mettre en attente le \emph{thread} courant jusqu'à la mort du \emph{thread} \mintinline{c}|thread_id|. Cette
fonction renvoie \mintinline{c}|false| si le \emph{thread} \mintinline{c}|thread_id| n'existe pas. Elle renvoie \mintinline{c}|true| sinon.\\

\textbf{Question 2~:} compléter le programme de la figure \ref{fig:prog2} (fichier \verb+main.c+) afin que l'affichage résultant de l'exécution de ce programme soit~:
\begin{minted}{c}
fct2
fct1
fin du main
\end{minted}

\begin{figure}[!htpb]
\begin{minted}{c}
    #include <stdio.h>
    #include "thread.h"

    void fct1(long t)
    {
      printf("fct1\n");
    }

    void fct2()
    {
      printf("fct2\n");
    }

    int main()
    {
      int t2 = create_thread((long)fct2, DEFAULT_PRIORITY, "thread2", 0);
      int t1 = create_thread((long)fct1, DEFAULT_PRIORITY, "thread1", 1, t2);
      resume(t1);
      resume(t2);
      printf("fin du main\n");
      return 0;
    }
\end{minted}
\caption{Programme à compléter pour obtenir l'affichage décrit dans le texte}
\label{fig:prog2}
\end{figure}

\textbf{Question 3~:} compléter la procédure \mintinline{c}|yield| et la fonction \mintinline{c}|join| (fichier \verb+yield.c+ et \verb+join.c+).

\subsection{Sémaphores}

\subsubsection{Principe}

Les \emph{sémaphores} vont permettre de synchroniser les \emph{threads} de maniére plus fine. Un sémaphore possède un compteur et quatre opérations~:\\

\begin{itemize}

\item l'opération de création, qui permet de créer un sémaphore en initialisant son compteur à la valeur désirée~;\\

\item l'opération de destruction, qui permet de détruire le sémaphore~;\\

\item l'opération \emph{P} décrémente le compteur et~:
\begin{itemize}
\item si le compteur est négatif, le \emph{thread} courant est mis dans l'état \mintinline{c}|WAITING|, il est placé dans la file d'attente associée au
sémaphore et un autre \emph{thread} prend la main~;
\item si le compteur est positif ou nul, le \emph{thread} courant garde la main.\\
\end{itemize}

\item l'opération \emph{V} incrémente le compteur et~:
\begin{itemize}
\item si le compteur est négatif ou nul, un \emph{thread} de la file d'attente du sémaphore est retiré de celle-ci et placé dans
la \mintinline{c}|ready_list|~;
\item si le compteur est strictement positif, aucune autre action est nécessaire.\\
\end{itemize} 

\end{itemize}

Dans notre gestionnaire de \emph{threads}, l'opération de création d'un \emph{semaphore} est la fonction \\
\mintinline{c}|int create_semaphore(int count)|,
elle prend en paramètre la valeur initiale du compteur et retourne l'identifiant du sémaphore ou $-1$ s'il n'y a plus de sémaphore disponible.\\

L'opération de destruction d'un sémaphore est la fonction 
\mintinline{c}|bool destroy_semaphore(int sem)|. Elle prend en paramètre l'identifiant d'un \emph{sémaphore}
et le détruit. Cette fonction retourne \mintinline{c}|true| si l'identifiant est valide et \mintinline{c}|false| sinon.\\

L'opération \emph{P} est la fonction \mintinline{c}|bool p(int sem)|. Elle prend en paramètre l'identifiant d'un 
sémaphore, elle renvoie \mintinline{c}|false| si le numéro de \emph{sémaphore} passé en paramètre n'est pas valide et renvoie \mintinline{c}|true| sinon.\\

L'opération \emph{V} est la fonction \mintinline{c}|bool v(int sem)|. Elle prend en paramètre l'identifiant d'un 
sémaphore, elle renvoie \mintinline{c}|false| si le numéro de \emph{sémaphore} passé en paramètre n'est pas valide et renvoie \mintinline{c}|true| sinon.\\

\begin{figure}[!htpb]
\begin{minted}{c}
#include <stdio.h>
#include "thread.h"
#include "semaphore.h"

const unsigned int N = 1000000000;
volatile int n = 0;

void fct1()
{
  for (unsigned int i = 0; i < N; i++) n++;
}

void fct2()
{
  for (unsigned int i = 0; i < N; i++) n--;
}

int main()
{
  int t1 = create_thread((long)fct1, DEFAULT_PRIORITY, "thread1", 0);
  int t2 = create_thread((long)fct2, DEFAULT_PRIORITY, "thread2", 0);
  resume(t1);
  resume(t2);
  join(t1);
  join(t2);
  printf("%d\n",n);
  return 0;
}
\end{minted}
\caption{Programme à compléter pour protéger l'accès à la variable globale}
\label{fig:prog3}
\end{figure}

\textbf{Question 4~:} soit le programme de la figure \ref{fig:prog3} (fichier \verb+main.c+). Si on exécute ce programme, la valeur de $n$ affichée
dans le \verb+main+ pourra être différente de $0$. Pourquoi~?\\
On voudrait protéger l'accès à la variable globale \mintinline{c}|n|. Ajouter
un sémaphore permettant de résoudre ce problème.\\

\begin{figure}[!htpb]
\begin{minted}{c}
    #include <stdio.h>
    #include "thread.h"
    #include "semaphore.h"

    void fct1()
    {
      printf("avant synchronisation\n");
      printf("après synchronisation\n");
    }

    void fct2()
    {
      printf("avant synchronisation\n");
      printf("après synchronisation\n");
    }

    void fct3()
    {
      printf("avant synchronisation\n");
      printf("après synchronisation\n");
    }

    int main()
    {
      resume(create_thread((long)fct1, DEFAULT_PRIORITY, "thread1", 0));
      resume(create_thread((long)fct2, DEFAULT_PRIORITY, "thread2", 0));
      resume(create_thread((long)fct3, DEFAULT_PRIORITY, "thread3", 0));
      return 0;
    }
\end{minted}
\caption{Programme à compléter pour obtenir un rendez-vous}
\label{fig:prog4}
\end{figure}

\textbf{Question 5~:} soit le programme de la figure \ref{fig:prog4} (fichier \verb+main.c+). On voudrait faire en sorte que les trois messages \verb+avant synchronisation+ s'affichent avant les trois messages \verb+après synchronisation+. Réaliser ceci grâce à des sémaphores.

\subsubsection{Structures de données}

Les structures de données associées à un \emph{sémaphore} sont les suivantes (fichier \verb+semaphore.h+)~:

\begin{minted}{c}
typedef struct {
  int state;
  int count;
  list waiting_list;
} semaphore;
\end{minted}

\begin{itemize}
\item \mintinline{c}|state| indique si le sémaphore est utilisé ou non (\mintinline{c}|SFREE| ou \mintinline{c}|SUSED|)~;\\

\item \mintinline{c}|count| est le compteur associé au sémaphore~;\\

\item \mintinline{c}|waiting_list| est la liste des \emph{threads} en attente sur ce sémaphore.\\

\end{itemize}

Le tableau \mintinline{c}|extern semaphore semaphore_table[];| 
contient les informations sur les \emph{semaphores}. Le numéro d'un \emph{semaphore} correspond à son indice dans ce tableau.\\

\textbf{Question 6~:} implémenter la fonction \mintinline{c}|bool p(int sem)| (fichier \verb+semaphore.c+) qui réalise l'opération \emph{P} décrite ci-dessus.\\

\textbf{Question 7~:} implémenter la fonction \mintinline{c}|bool v(int sem)| (fichier \verb+semaphore.c+) qui réalise l'opération \emph{V} décrite ci-dessus.\\

\section{Communication entre \emph{threads}}

Nous allons proposer un mécanisme simple de communication entre \emph{threads} grâce aux fonctions \mintinline{c}|send| et \mintinline{c}|receive| dont
le fonctionnement est décrit ci-dessous~:\\

\begin{itemize}

\item \mintinline{c}|int receive()|. Lorsque le \emph{thread} courant fait cet appel, deux cas de figures peuvent se passer~:
\begin{itemize}
\item un message lui a déjà été envoyé par un autre \emph{thread} et cet appel retourne immédiatement le message reçu après avoir
mis à \mintinline{c}|false| le champ \mintinline{c}|received| du contexte du \emph{thread} courant~;
\item aucun message ne lui a encore été envoyé, et le \emph{thread} courant est alors mis en attente dans l'état \mintinline{c}|RECEIVE|.\\
\end{itemize}

\item \mintinline{c}|bool send(int thread_id, int message)| permet d'envoyer au \emph{thread} \mintinline{c}|thread_id| l'entier \mintinline{c}|message|.
L'entier
\verb+message+ est copié dans le champ \mintinline{c}|message| du contexte associé au \emph{thread} \mintinline{c}|thread_id| et le champ
\mintinline{c}|received| de cette structure est mis à \mintinline{c}|true|. Si le \emph{thread} \mintinline{c}|thread_id| était en attente de recevoir un message, 
il est alors remis dans la \mintinline{c}|ready_list|. La fonction retourne alors la valeur \mintinline{c}|true|.\\
Notons que si un message
n'avait pas encore été lu par le \emph{thread} \mintinline{c}|thread_id|, ou si l'identifiant \mintinline{c}|thread_id| n'existe pas, la fonction ne fait que renvoyer
\mintinline{c}|false|.\\

\end{itemize}

\begin{figure}[!htpb]
\begin{minted}{c}
    #include <stdio.h>
    #include "thread.h"
    #include "message.h"

    int t1, t2;

    void fct1()
    {
    }

    void fct2()
    {
    }

    int main()
    {
      t1 = create_thread((long)fct1, DEFAULT_PRIORITY, "thread1", 0);
      t2 = create_thread((long)fct2, DEFAULT_PRIORITY, "thread2", 0);
      resume(t1);
      resume(t2);
      return 0;
    }
\end{minted}
\caption{Programme à compléter afin d'envoyer des messages entre deux \emph{threads}}
\label{fig:prog5}
\end{figure}


\textbf{Question 8~:} soit le programme de la figure \ref{fig:prog5} (fichier \verb+main.c+). 
On veut que le \emph{thread1} envoie $10$ messages (les entiers de $0$ à $9$) au \emph{thread2} qui les affichera. 
Réaliser ceci grâce aux primitives \mintinline{c}|send| et \mintinline{c}|receive|.\\

\textbf{Question 9~:} implémenter les fonctions \mintinline{c}|send| et \mintinline{c}|receive| (fichier \verb+message.c+).

\section{Endormir un \emph{thread}}

La fonction \mintinline{c}|bool sleep(int n)| (fichier \verb+sleep.h+)
permet d'endormir le \emph{thread} courant pendant $(n\times 10)ms$. Cette fonction retourne \mintinline{c}|true| si $n > 0$ et \mintinline{c}|false| sinon.
Un \emph{thread} mis en sommeil sera mis dans l'état \mintinline{c}|ASLEEP|.\\

La procédure \mintinline{c}|void wake_up()| (fichier \verb+sleep.h+) permet de réveiller le ou les \emph{threads} dont le temps de mise en sommeil
est terminé (cette procédure est appelée dans le traitant d'interruption \mintinline{c}|clock_handler| à chaque appel de celui-ci).\\

\textbf{Question 10~:} implémenter puis tester les fonctions \mintinline{c}|sleep| et \mintinline{c}|wake_up| (fichier \verb+sleep.c+).

\end{document}
